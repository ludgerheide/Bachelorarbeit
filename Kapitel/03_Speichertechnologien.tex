\chapter{Speichertechnologien} %Zuerst
Nach den Ladesystemen sollen hier nun die Speichertechnologien betrachtet werden. Sie lassen sich in mechanische, elektrische und elektrochemische Speicher aufteilen. In diesem Kapitel werden die Wirkprinzipien der in Bussen verwendeten Speichertechnologien erklärt.\\
\section{Mechanisch – Schwungradspeicher} %TODO: Quellen, warum endete die Gyrobuserprobung
Mechanische Energiespeicher für Fahrzeuge arbeiten mit komprimierter Luft oder Schwungrädern. Es gibt Prototypen reiner Pressluftantrieben in kleineren Fahrzeugen, in Bussen werden sie jedoch nur als Teil eines Hybridantriebs eingesetzt und hier nicht weiter betrachtet.\cite{Sebastian-Naumann:2014}. Im Schwungradspeicher wird die elektrische Energie in der Rotationsenergie eines Schwungrades gespeichert, das sich mit sehr hoher Geschwindigkeit dreht. Die Energieübertragung erfolgt durch eine elektrische Motor- und Generatoreinheit. Moderne Schwungräder werden aus gewickelten Karbonfasern hergestellt und in Vakuumgehäusen magnetisch gelagert. Im Falle eines berstenden Schwungrades muss das Gehäuse die gesamte Energie innerhalb von Sekundenbruchteilen aufnehmen, ohne selbst zu bersten. Dies erfordert sehr schwere Gehäuse, die die spezifische Energie und Leistung eines tatsächlichen Systems stark reduzieren. Der Schwungradspeicher wurde in den fünfziger Jahren im Gyrobus im schweizerischen Yverdon auf einer acht Kilometer langen Linie erprobt. Die acht Kilometer lange Strecke wurde erfolgreich zurückgelegt, die damalige Technologie war jedoch sehr wartungsaufwändig. Aktuell wird der Schwungradspeicher nur als Teil eines hybriden Antriebsstrangs eingesetzt.
\section{Elektrisch – Kondensator} %TODO: Quellen
Der Kondensator ist ein rein elektrischer Energiespeicher. Im klassischen Plattenkondensator werden zwei durch ein Dieelektrikum getrennte Platten elektrisch aufgeladen und die Ladung kann später wieder in Strom umgewandelt werden. Kondensatoren haben eine hohe spezifische Leistung, aber ihre spezifische Energie reicht nicht aus. In Bussen werden sogenannte Superkondensatoren verwendet, die statt eines festen Dieelektrikums eine polare Flüssigkeit verwenden und mithilfe des sogenannten Doppelschichteffektes und der Pseudokapazität weit höhere Energiedichten erreichn. In Shanghai werden Busse mit dieser Technologie seit 2008 im Linienverkehr eingesetzt.
\section{Chemisch}
\cite{Lajunen20141}
\section{Bewertungskriterien} %TODO: Subsections
\section{Betrachtete Systeme} %TODO: Subsections
\section{Vergleichstabelle}   %TODO: In den Anhang?