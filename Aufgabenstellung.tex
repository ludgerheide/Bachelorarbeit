\documentclass{scrartcl}

%%%%% BEGINN PACKAGES %%%%%

%Packages, die für die deutsch Sprache erforderlich sind
\usepackage[utf8]{inputenc}
\usepackage[T1]{fontenc}
\usepackage{lmodern}
\usepackage{ngerman}

%Packages für Graphik
\usepackage{graphicx}

%Package, damit Bibtex-URL klappt
\usepackage{url}

%Package für Hyperlinks im PDF
\usepackage{hyperref}

%Package für Tabellen mit avriabler Breite
\usepackage{tabularx}

%Noch schönere Typographie
\usepackage{microtype}

%%%%% ENDE PACKAGES %%%%%
\title{Lade- und Speichertechnologien für Elektrobusse}
\subtitle{Bachelorarbeit}
\date{\today}
\author{Ludger Heide}



\begin{document}
\pagenumbering{gobble}
%%%%% BEGINN FRONT MATTER %%%%%

\maketitle

%%%%% ENDE FRONT MATTER %%%%%

%%%%% BEGINN INHALT %%%%%
\section*{Problemstellung und Zielsetzung}
Eine der wichtigsten Fragen beim Einsatz von batteriebetriebenen Stadtbussen ist die Frage, wie der Bus aufgeladen wird und wie die Energie im Bus gespeichert wird. Inzwischen sind diverse Ladesysteme weit genug entwickelt, um im realen Betrieb eingesetzt zu werden. Parallel werden neue Forschungsergebnisse im Bereich der Batterietechnologie schnell industriell umgesetzt.\\
In dieser Bachelorarbeit werden Ladesysteme und Speichertechnologien jeweils miteinander verglichen. Die Systeme werden hierbei aus technischer Sicht betrachtet. Relevant sind zum Beipiel mechanische, elektrische und sicherheitstechnische Parameter, wirtschaftliche und politische Aspekte werden nicht explizit betrachtet. Es werden Ladesysteme verglichen, die bereits konkret existieren und im normalen Verkehrsbetrieb erpropt wurden oder werden. Auch bei den Speichertechnologien sollen Technologien verglichen werden, die jetzt oder in naher Zukunft im industriellen Maßsstab produziert werden können.\\
Ziel ist es die Effizienz verschiedener Kombinationen von Ladetechnologie, Speichertechnologie und Ladestrategie auf verschiedenen Linien vergleichen zu können.

\section*{Grundsätzliche Vorgehensweise}
\begin{enumerate}
	\item Recherche über existierende Ladesysteme und Speichertechnologien
	\item Systematische Festlegung und Ermittlung der technischen Parameter
	\item Verlgeich von Technologiekombinationen auf verschiedenen Strecken
\end{enumerate}
%%%%% ENDE INHALT %%%%%

%%%%% BEGINN BACK MATTER %%%%%

%%%%% ENDE BACK MATTER %%%%%

\end{document}