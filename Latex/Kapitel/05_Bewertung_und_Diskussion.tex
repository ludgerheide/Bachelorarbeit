\chapter{Bewertung und Diskussion} %Kapitelüberschriften
In den vorigen Kapiteln wurden technische Daten zu Ladesystemen, Speichersystemen und deren Kombinationen erhoben und berechnet. In diesem Kapitel wird nun eine gewichtete Bewertung dieser Kriterien erstellt, um Sinnhaftigkeit der verschiedenen Gesamtsysteme betrachten zu können.

Die Systeme werden aus rein technischer Sicht betrachtet, wirtschaftliche Aspekte wurden ignoriert. Von daher ist diese Bewertung nur im Zusammenhang mit einer Betrachtung aus wirtschaftlicher Sicht für Busbetreiber relevant. Systeme, die hier als technisch sinnlos identifiziert wurden können bei einer folgenden wirtschaftlichen Betrachtung ausgeblendet werden.

\section{Gewichtung der Kritereien}
Die Auswahl und Gewichtung der Bewertungskriterien basiert auf der Masterarbeit von Thomas Pannwitz\marginpar{Formulierung}, in der eine Umfrage unter 17 Busbetreibern über ihre Entscheidungskriterien für die Busbeschaffung durchgeführt wurde. Die fünf wichtigsten Anforderungen der Busbetreiber waren~\cite{pannwitz2014}[S. 24f]:
\begin{itemize}
	\item hohe technische Zuverlässigkeit des Fahrzeugs
	\item geringe Störanfälligkeit gegenüber Vandalismus
	\item Verfügbarkeit von elektrischen Niederflurbussen
	\item Vermeidung von $CO_2$-Emissionen
	\item effizienter Einsatz unter extrem Umweltbedingungen
\end{itemize}

Auf Basis dieser Kriterien wurde die in Tabelle \ref{tab_bewertungskriterien} dargestellte Zuordnung der technischen Daten zu den verschiedenen Kategoerien durchgeführt\marginpar{Ausfüllen, Gewichtung!?}.



\begin{table}
	\centering
	\begin{tabularx}{\linewidth}{Xrl}
		\toprule
		Kriterium                        &                   & Gewichtung (\%) \\ \midrule
		\textbf{Zuverlässigkeit}                   & $\Sigma$          & 25                    \\
		Fahrzeugkilometer, Teststrecke   &                   & 5                     \\
		Fahrzeugkilometer, Linienbetrieb &                   & 15                    \\
		Erfahrungen im Probebetrieb      &                   & 5                     \\ \midrule
		\textbf{Sicherheit gegenüber Vandalismus}              & $\Sigma$          & 25                    \\
		Zugänglichkeit fahrzeugseitig    &                   & 10                    \\
		Zugänglichkeit stationsseitig    &                   & 15                    \\ \midrule
		\textbf{Komplexität Betrieb}     & $\Sigma$          & 25                    \\
		Freiheitsgrade gesamt            &                   & 5                     \\
		Positionierungsgenauigkeit       &                   & 10                    \\ \midrule
		\textbf{Komplexität Aufbau}      & $\Sigma$          & 10                    \\
		Volumen im Fahrzeug              &                   & 5                     \\
		Aufwand für Bau der Ladestation  &                   & 5                     \\ \midrule
		\textbf{Effizienz}               & $\Sigma$          & 15                    \\
		Energieverbrauch                 &                   & 9                     \\
		Ladedauer pro Betriebsstunde     &                   & 2                     \\ \bottomrule
		                                 & $\Sigma_{gesamt}$ & 100
	\end{tabularx}
	\label{tab_bewertungskriterien}
	\caption{Gewichtung der Bewertungskriterien der Gesamtlösungen}
\end{table} 

\section{title}

%TODO : Am Anfang alles exakt, dann Vereinfachung mit hinweis darfauf, das exaktheit verloren

%TODO: Mehrere Tabellen, erst Rohdaten, dann mit Scorewerten, dann gewichteten

%TODO: Warum habe ich keine binäre Dominanzmatrix verwendet?


\section{Fazit}
\section{Ausblick}