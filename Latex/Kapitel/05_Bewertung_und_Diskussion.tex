\chapter{Bewertung und Diskussion} %Kapitelüberschriften
In den vorigen Kapiteln wurden technische Daten zu Ladesystemen, Speichersystemen und deren Kombinationen erhoben und berechnet. In diesem Kapitel wird nun eine gewichtete Bewertung dieser Kriterien erstellt, um Vor- und Nachteile der verschiedenen Gesamtsysteme betrachten zu können.

Die Systeme werden aus rein technischer Sicht betrachtet, wirtschaftliche Aspekte wurden ignoriert. Von daher ist diese Bewertung nur im Zusammenhang mit einer Betrachtung aus wirtschaftlicher Sicht für Busbetreiber relevant. Systeme, die hier als technisch sinnlos identifiziert wurden können bei einer folgenden wirtschaftlichen Betrachtung ausgeblendet werden.

\section{Gewichtete Bewertung}
Im vorigen Kapitel wurden die Werte 14 verschiedener Parameter für alle Kombinationen von fünf Batterien, insgesamt fünf Ladesystemen und zwei Buslinien berechnet. So sind insgesamt 700 Datenpunkte entstanden, die nun in einer verständlichen Form dargestellt werden müssen. Dazu  

\subsection{Gewichtung der Kritereien}
Die Auswahl und Gewichtung der Bewertungskriterien basiert auf der Masterarbeit von Thomas Pannwitz, in der eine Umfrage unter 17 Busbetreibern über ihre Entscheidungskriterien für die Busbeschaffung durchgeführt wurde. Die fünf wichtigsten Anforderungen der Busbetreiber waren~\cite{pannwitz2014}[S. 24f]:
\begin{enumerate}
	\item hohe technische Zuverlässigkeit des Fahrzeugs
	\item geringe Störanfälligkeit der Ladestation
	\item Verfügbarkeit von elektrischen Niederflurbussen
	\item Vermeidung von $CO_2$-Emissionen
	\item effizienter Einsatz unter extrem Umweltbedingungen
\end{enumerate}

Auf Basis dieser Kriterien wurde das in Tabelle \ref{tab_bewertungskriterien} dargestellte Gewichtungsschema konstruiert.
%TODO: Remove
\begin{quotation}
	
	\textbf{Notizen zu Tabelle \ref{tab_bewertungskriterien}}
	\begin{itemize}
		\item Masse $\rightarrow$ Zuverlässigkeit (wegen Verschleiß)?
		\item Exponierte Leiter wo?
		\item Volumen Ladestation konnte nirgendwo untergebracht werden
	\end{itemize}
\end{quotation}

\begin{table}
	\centering
	\begin{tabularx}{\linewidth}{Xrl}
		\toprule
		Kriterium                                                &                   & Gewichtung (\%) \\ \midrule
		\textbf{Zuverlässigkeit Fahrzeug}                        &                   &  \\
		Ladezyklen pro Kilometer                                 &                   & 8               \\
		Freiheitsgrade Ladesystem Fahrzeug                       &                   & 8               \\
		Positionierungsgenauigkeit (Fläche)                      &                   & 6               \\
		Positionierungsgenauigkeit (Winkel)                      &                   & 3               \\
		Zugänglichkeit Schnittstelle Fahrzeug                    &                   & 3               \\
		Masse Ladesystem + Batterie                              &                   & 2               \\ \midrule
		                                                         &          $\Sigma$ & \textbf{30}     \\
		\textbf{Störanfälligkeit der Ladestation}                &                   &  \\
		Anteil Ladezeit an Gesamtzeit                            &                   & 6,25             \\
		Exponierte Leiter                                        &                   & 6,25             \\
		Freiheitsgrade Ladestation                               &                   & 6,25             \\
		Zungänglichkeit Ladestation                              &                   & 6,25 \\ \midrule
		                                                         &          $\Sigma$ & \textbf{25}     \\
		\textbf{Verfügbarkeit von elektrischen Niederflurbussen} &                   &  \\
		Volumen Ladesystem (feste Position)                      &                   & 10              \\
		Volumen Ladesystem (freie Position) + Volumen Batterie   &                   & 10              \\ \midrule
		                                                         &          $\Sigma$ & \textbf{20}     \\
		\textbf{Vermeidung von CO\textsubscript{2}-Emissionen}   &                   &  \\
		Energieverbrauch                                         &                   & 15              \\ \midrule
		                                                         &          $\Sigma$ & \textbf{15}     \\
		\textbf{Einsatz unter extrem Umweltbedingungen}          &                   &  \\
		Kühlluftmasse                                            &                   & 10              \\
		                                                         &          $\Sigma$ & \textbf{10}     \\ \midrule
		                                                         & $\Sigma_{gesamt}$ & 100
	\end{tabularx}
	\label{tab_bewertungskriterien}
	\caption{Gewichtung der Bewertungskriterien der Gesamtlösungen}
\end{table} 

\subsection{Zuweisung von Scorewerten}

%TODO : Am Anfang alles exakt, dann Vereinfachung mit hinweis darfauf, das exaktheit verloren

%TODO: Mehrere Tabellen, erst Rohdaten, dann mit Scorewerten, dann gewichteten

%TODO: Warum habe ich keine binäre Dominanzmatrix verwendet?


\section{Fazit}
\section{Ausblick}