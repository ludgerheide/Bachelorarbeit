\chapter{Bewertung und Diskussion} %Kapitelüberschriften
In den vorigen Kapiteln wurden technische Daten zu Ladesystemen, Speichersystemen und deren Kombinationen erhoben und berechnet. In diesem Kapitel wird nun eine gewichtete Bewertung dieser Kriterien erstellt, um Vor- und Nachteile der verschiedenen Gesamtsysteme betrachten zu können.

Die Systeme werden aus rein technischer Sicht betrachtet, wirtschaftliche Aspekte wurden ignoriert. Von daher ist diese Bewertung nur im Zusammenhang mit einer Betrachtung aus wirtschaftlicher Sicht für Busbetreiber relevant. Systeme, die hier als technisch sinnlos identifiziert wurden können bei einer folgenden wirtschaftlichen Betrachtung ausgeblendet werden.

\section{Gewichtete Bewertung}
Im vorigen Kapitel wurden die Werte 14 verschiedener Parameter für alle Kombinationen von fünf Batterien, insgesamt fünf Ladesystemen und zwei Buslinien berechnet. So sind 700 Datenpunkte entstanden. Diese Menge muss nun mit einem geeigneten Verfahren auf wenige Kennwerte reduziert werden, mit denen die Eignung einer bestimmten Kombination von Ladestrategie, Ladesystem und Speichertechnologie für eine bestimmte Strecke bestimmt werden kann.

Dazu wurde die Methode der gewichteten technischen Bewertung nach VDI~2225~Blatt~3~\cite{vdi:2225} gewählt. Diese besteht aus drei Phasen:
\begin{enumerate}
	\item Festlegung der Parameter und Mindestanforderungen
	\item Zuordnung von Punktzahlen zu Leistungsmerkmalen
	\item Gewichtung der Punktzahlen
	\item Berechnung der gewichteten Mittelwerte
\end{enumerate}

Die Festlegung der Parameter ist bereits in den Abschnitten \ref{parameter} und \ref{simErgebnisse}. Die Erfüllung der Mindestanforderungen hinsichtlich Stromstärke und Energie wurden durch die Optimierung der Batteriegröße sichergestellt. Um bei einer maximalen Fahrzeugmasse von 18 Tonnen und einem Leergewicht von 13 Tonnen mindestens 50\marginpar{Zahlenwerte} Passagiere transportieren zu können, darf das Gewicht von Batterie und Ladesystem 1,6 Tonnen nicht überschreiten.

\paragraph{Digitaler Anhang} In Anhang \ref{an_Digital} ist die zur Berechnung verwendete Tabelle enthalten. Sie ist unter dem Namen \texttt{Auswertung.ods} zu finden und kann mit \textsc{LibreOffice} geöffnet werden.

\subsection{Zuordnung von Punktzahlen}
Zur Bewertung werden die Kriterien aus Abschnitt \ref{parameter} und \ref{simErgebnisse} verwendet. Die Massen von Ladesystem und Batterie sowie das Batterievolumen und das freie Volumen des Ladesystems werden zu einem Gesamtvolumen und einer Gesamtmasse zusammengefasst.

Die Zuordnung der Punkte erfolgt anhand einer subjektiven Bewertung der Güte eines bestimmten Wertebereichs. Es wird darauf geachtet, das der Durchschnittswert der Punktzahlen der technisch akzeptablen Lösungen bei 2,5 liegt. Die Bewertungen der Punkte sind in Tabelle \ref{tabPunkte} dargestellt.

\begin{table} \centering
	\begin{tabular}{ll}
		\toprule
		Bewertung & Punktzahl \\ \midrule
		0 & unbefriedigend \\
		1 & gerade noch tragbar \\
		2 & ausreichen \\
		3 & gut \\
		4 & sehr gut \\ \bottomrule
	\end{tabular}
	\caption[Bewertung der Punktzahlen]{Bewertung der Punktzahlen nach VDI~2225 Blatt~3~\cite{vdi:2225}[S. 5]}
	\label{tabPunkte}
\end{table}

Auf Basis dieses Schemas wurden den berechneten Werten die in Tabelle \ref{tab_punktzahlen} dargestellten Punktzahlen zugeordnet.

%TODO: Grad in die Mitte, statt an den Rand?
\begin{table}
	\centering
	\begin{tabulary}{\linewidth}{LLrrrr}
		\toprule
		Kriterium                                                & Einheit          &         1P &      2P &        3P &           4P \\ \midrule
		\textbf{Zuverlässigkeit Fahrzeug}                        &                  &            &         &           &  \\
		Ladezyklen pro Kilometer                                 & $\frac{1}{km}$   & $\ge 0,03$ & $<0,03$ &   $<0,01$ &     $<0,005$ \\
		Freiheitsgrade Ladesystem Fahrzeug                       & 1                &       $>2$ &       2 &         1 &            0 \\
		Positionierungsgenauigkeit (Fläche)                      & $m^2$            &       $<3$ &    $<6$ &      $<9$ &      $\ge 9$ \\
		Positionierungsgenauigkeit (Winkel)                      & $^\circ$         &       $<5$ &     --- &       --- &    $\ge 360$ \\
		Zugänglichkeit Schnittstelle Fahrzeug                    & ---              &        --- &   Seite &     Boden &         Dach \\
		Masse Ladesystem + Batterie                              & kg               & $\ge 1250$ & $<1250$ &    $<750$ &       $<250$ \\ \midrule
		\textbf{Störanfälligkeit der Ladestation}                &                  &            &         &           &  \\
		Anteil Ladezeit an Gesamtzeit                            & 1                &  $\ge 0,4$ &  $<0,4$ &    $<0,2$ &       $<0,1$ \\
		Exponierte Leiter                                        & ---              &      immer &     --- & nur Laden &          nie \\
		Zugänglichkeit Ladestation                               & ---              &  ebenerdig &  erhöht & unterflur & unzugänglich \\ \midrule
		\textbf{Verfügbarkeit von elektrischen Niederflurbussen} &                  &            &         &           &  \\
		Volumen Ladesystem (feste Position)                      & l                & $\ge 1000$ & $<1000$ &    $<500$ &       $<250$ \\
		Volumen Ladesystem freie Position + Volumen Batterie     & l                &  $\ge 450$ &  $<450$ &    $<300$ &       $<150$ \\ \midrule
		\textbf{Vermeidung von CO\textsubscript{2}-Emissionen}   &                  &            &         &           &  \\
		Energieverbrauch                                         & $\frac{kWh}{km}$ &  $\ge 1,5$ &  $<1,5$ &    $<1,2$ &       $<0,9$ \\ \midrule
		\textbf{Einsatz unter extrem Umweltbedingungen}          &                  &            &         &           &  \\
		Kühlluftmasse                                            & $\frac{g}{s}$     & $\ge 5000$ & $<5000$ &   $<3000$ &      $<1000$ \\ \bottomrule
	\end{tabulary}
	\caption{Gewichtung der Bewertungskriterien der Gesamtlösungen}
	\label{tab_punktzahlen}
\end{table} 

\subsection{Gewichtung der Punktzahlen}
Die Auswahl und Gewichtung der Bewertungskriterien basiert auf der Masterarbeit von Thomas Pannwitz, in der eine Umfrage unter 17 Busbetreibern über ihre Entscheidungskriterien für die Busbeschaffung durchgeführt wurde. Die fünf wichtigsten Anforderungen der Busbetreiber waren~\cite{pannwitz2014}[S. 24f]:
\begin{enumerate}
	\item hohe technische Zuverlässigkeit des Fahrzeugs
	\item geringe Störanfälligkeit der Ladestation
	\item Verfügbarkeit von elektrischen Niederflurbussen
	\item Vermeidung von $CO_2$-Emissionen
	\item effizienter Einsatz unter extrem Umweltbedingungen
\end{enumerate}

Auf Basis dieser Kriterien wurde das in Tabelle \ref{tab_bewertungskriterien} dargestellte Gewichtungsschema konstruiert.
%TODO: Remove
\begin{quotation}	
	\textbf{Notizen zu Tabellen \ref{tab_bewertungskriterien} und \ref{tab_punktzahlen}}
	\begin{itemize}
		\item Masse $\rightarrow$ Zuverlässigkeit (wegen Verschleiß)?
		\item Volumen Ladestation konnte nirgendwo untergebracht werden
		\item Volumen Fahrzeug fest steht nicht unbedingt im Zusammenhang zu Niederflurbussen
	\end{itemize}
\end{quotation}

\begin{table}
	\centering
	\begin{tabularx}{\linewidth}{Xrl}
		\toprule
		Kriterium                                                &                   & Gewichtung (\%) \\ \midrule
		\textbf{Zuverlässigkeit Fahrzeug}                        &                   &  \\
		Ladezyklen pro Kilometer                                 &                   & 8               \\
		Freiheitsgrade Ladesystem Fahrzeug                       &                   & 8               \\
		Masse Ladesystem + Batterie                              &                   & 6               \\
		Positionierungsgenauigkeit (Winkel)                      &                   & 3               \\
		Zugänglichkeit Schnittstelle Fahrzeug                    &                   & 3               \\
		Positionierungsgenauigkeit (Fläche)                      &                   & 4               \\ \midrule
		                                                         &          $\Sigma$ & \textbf{30}     \\
		\textbf{Störanfälligkeit der Ladestation}                &                   &  \\
		Anteil Ladezeit an Gesamtzeit                            &                   & 15             \\
		Exponierte Leiter                                        &                   & 5             \\
		Zungänglichkeit Ladestation                              &                   & 5 \\ \midrule
		                                                         &          $\Sigma$ & \textbf{25}     \\
		\textbf{Verfügbarkeit von elektrischen Niederflurbussen} &                   &  \\
		Volumen Ladesystem (feste Position)                      &                   & 10              \\
		Volumen Ladesystem (freie Position) + Volumen Batterie   &                   & 10              \\ \midrule
		                                                         &          $\Sigma$ & \textbf{20}     \\
		\textbf{Vermeidung von CO\textsubscript{2}-Emissionen}   &                   &  \\
		Energieverbrauch                                         &                   & 15              \\ \midrule
		                                                         &          $\Sigma$ & \textbf{15}     \\
		\textbf{Einsatz unter extrem Umweltbedingungen}          &                   &  \\
		Kühlluftmasse                                            &                   & 10              \\ \midrule
		                                                         &          $\Sigma$ & \textbf{10}     \\ \midrule
		                                                         & $\Sigma_{gesamt}$ & 100 \\ \bottomrule
	\end{tabularx}
	\caption{Gewichtung der Bewertungskriterien der Gesamtlösungen}
	\label{tab_bewertungskriterien}
\end{table} 

\subsection{Ergebnisse}
Die Ergebnisse der Bewertung sind in Tabelle \ref{tab_ergebnisse} zu sehen. Neben dem gewichteten Mittelwert zu jeder Kombination von Ladesystem und Speichertechnologie sind Durchschnittswerte für ein Ladesystem bzw. eine Speichertechnologie und Linie berechnet worden.

\begin{table}
	\centering
	\begin{tabular}{lllllll}
		                       \multicolumn{7}{c}{\textbf{Buslinie 204}}                         \\ \toprule
		                      & 40kW & 200kW & 375kW & Swap & Nachtladen  & Zeilen-$\varnothing$ \\ \midrule
		18650                 & 2,83 & 2,43  & 2,09  & 2,95 & 3,05        & 2,67                 \\
		LiTiO                 & 3,07 & 2,77  & 2,34  & 2,79 & \emph{0,00} & 2,19                 \\
		LFP-HE                & 2,34 & 2,30  & 1,81  & 2,31 & 2,64        & 2,28                 \\
		LFP-HP                & 2,39 & 1,99  & 1,71  & 2,51 & 2,38        & 2,20                 \\
		Spalten-$\varnothing$ & 2,66 & 2,37  & 1,99  & 2,64 & 2,02        &  \\ \midrule
		                       \multicolumn{7}{c}{\textbf{Buslinie 192}}                         \\
		                      & 40kW & 200kW & 375kW & Swap & Nachtladen  & Zeilen-$\varnothing$ \\ \midrule
		18650                 & 2,77 & 2,58  & 2,09  & 2,95 & 3,07        & 2,69                 \\
		LiTiO                 & 3,07 & 2,77  & 2,34  & 2,94 & \emph{0,00} & 2,22                 \\
		LFP-HE                & 2,42 & 2,22  & 1,79  & 2,54 & 2,66        & 2,33                 \\
		LFP-HP                & 2,52 & 2,02  & 1,74  & 2,74 & \emph{0,00} & 1,80                 \\
		Spalten-$\varnothing$ & 2,70 & 2,40  & 1,99  & 2,79 & 1,43        &  \\ \bottomrule
	\end{tabular}
	\caption{Ergebnisse der gewichteten Bewertung}
	\label{tab_ergebnisse}
\end{table} 

\section{Diskussion} \marginpar{Lieber "`Interpretation"' als Titel?}
Die am besten bewertetn Kombinationen sind für beide Buslinien die Kombination von Lithium-Titanat-Batterie und konduktiv-manueller Aufladung sowie das Aufladen über Nacht mit Lithium-Schichtoxid-Batterien.

Die hohe Bewertung des konduktiv-manuellen Ladesystems lässt sich durch den Fokus auf Zuverlässigkeit und Kompaktheit erklären. Dieses Ladesystem ist vergleichsweise einfach und leicht, es besitzt keine automatisch bewegten Teile und stellt keine hohen Anforderungen an die Position des Busses. Die Ladezeit beträgt zwischen 25 und 30 Minuten nach jeder Tour und ist damit fast dreimal so lang wie die des induktiven Systems. Um die gleiche Taktfrequenz werden daher mit der langsameren Technologie mehr Fahrzeuge und Fahrer benötigt, die erhöhte Kosten mit sich bringen. Für eine genauere Analyse ist neben den Kosten sämtlicher Komponenten also auch eine Betrachtung der gewünschten Taktfrequenz nötig.

Die hohe Bewertung von Nachtladesystem widerspricht der anfänglichen Hypothese, das die benötigten scheren Batterien nicht praktikabel seien. Zwei der vier Batterien haben jedoch genug Kapazität, einen Betriebstag mit 11 Touren zu überstehen. Auch der erwartete höhere Energieaufwand ist nicht aufgetreten. Es wird sogar der niedrigste Energieverbrauch auf beiden Buslinien erreicht. Dies lässt sich durch die bessere Rekuperationsleistung der Batterien erklären. Die spezifische Rekuperationsleistung ist in der Simulation das begrenzende Element gewesen, eine größere Batterie kann mehr rekuperieren. Im Betrieb werden bei der Ladestrategie "`Nachtladung"' sogar weniger Fahrzeuge benötigt als bei jeder anderen Ladestrategie, da während des Betriebs tagsüber gar keine Ladezeit anfällt. Im Gegenzug sind die Fahrzeuge durch die großen Batterien in der Anschaffung erheblich teurer und durch das hohe Gewicht verschleißen Fahrzueg und Straße schneller. Auch hier ist eine Betrachtung der geforderten Taktfrequenz nötig, auch die Verschleißkosten der Straßen für verschiedene Fahrzeuggewichte sollten betrachtet werden.

Zur Abschätzung der Batterielebensdauer wurden die Ladezyklen pro Kilometer berechnet, die an der Batterie anfallen. Es fällt auf, das diese bei der Ladestrategie "`Gelegenheitsladung"' ca. um den Faktor fünf höher sind als bei Nachtladung. Auch die Zellenströme sind bei den großen Batterien der Ladestrategie "`Nachtladung"' weit niedriger. Bei Gelegenheitsladung werden die Batterien zwischen 70\% und 30\% Ladezustand betrieben, bei Nachtladung zwischen 90\% und 10\%. Welche Ladestrategie die bessere Lebensdauer mit sich bringt lässt sich schwer abschätzen und sollte weiter erforscht werden.

Das insgesamt am besten bewertete Ladesystem ist das Wechseln der Batterien durch Roboter in einem speziellen Gebäude. Es zeichnet sich durch eine "`konstante"' Ladezeit aus, da die Batterie im Bus nach dem sieben Minuten langen Wechselvorgang unabhängig von ihrer Größe "`voll"' ist. Daneben muss keine Ladeelektrik im Fahrzeug mitgeführt werden und die Batterien können außerhalb des Busses sehr effizient geladen werden. Allerdings ist die Ladestation sehr viel aufwändiger, da die Batterien  von Robotern entnommen und in Ladebuchten transportiert werden müssen. Von daher macht dieses System für den Einsatz von nur wenigen Elektrobussen keinen Sinn. Erst beim Einsatz einer großen Flotte wird die Investition in so ein aufwändigeres Ladesystem sinnvoll.

Das konduktiv-automatische Ladesystem mit 375 kW hat in jeder Kombination schlechter abgeschnitten als das induktive System mit 200k W. Aufgrund der begrenzten Ladeleistung der Batterien hat die höhere Ladeleistung keinen Vorteil, selbst der Lithium-Titanat-Akku erreicht eine maximale Ladeleistung von ca. 170 kW\marginpar{170 kW: Wert!}. Aufgrund der größeren und exponierteren Komponenten und der größeren Anzahl an Komponenten wird eine höhere Fehleranfälligkeit erwartet. Der ca. 5\% niedrigere Energieverbrauch reichte in der Bewertung nicht aus, um die Nachteile auszugleichen. Durch den Einsatz einer größeren Batterie könnte auch mit aktueller Technologie eine Ladezeit von 5 Minuten pro Tour erreicht werden, dies würde jedoch eine doppelt so große und teure Batterie erfordern.

Es wurde auch untersucht, ob Bleibatterien bei Gelegenheitsladung immer noch eine preiswerte alternative sind. Ergebnis ist, dass die gewählte Bleibatterie nicht aufgrund ihrer niedrigen spezifischen Energie zu schwer ist, sondern durch die spezifische Lade- und Entladeleistung begrenzt ist, weshalb ihr Einsatz keinen Sinn macht.

Alle anderen Batterien können die geforderte Leistung mit einer gewissen Reserve lieferen, die Rekuperationsleistung ist jedoch begrenzt. Hier bietet sich der Einsatz von Superkondensatoren an, um Leistungsspitzen beim Bremsen aufzunehmen und so die Energieeffizienz – und damit die erforderliche Batteriegröße – zu verbessern.

\section{Fazit}

\section{Ausblick}