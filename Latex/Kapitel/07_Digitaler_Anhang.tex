
\chapter{Digitaler Anhang}
\label{an_Digital}
Im digitalen Anhang sind folgende Dokumente enthalten:
\begin{itemize}
	\item Diese Arbeit als \textsc{PDF}, \textsc{HTML} und \textsc{ePub}
	\item Sämtliche digital verfügbaren Quellen als PDF
	\item Das verwendete Simulationsmodell (erfordert \textsc{Matlab} (ab R2012a) und \textsc{simulink} mit \textsc{simPowerSystems})
	\item Die Auswertungstabelle (erfordert \textsc{LibreOffice})
\end{itemize}

Der digitale Anhang ist sowohl auf der CD auf der letzten Seite als auch im \href{http://lht.no-ip.biz/share/public/Technologievergleich\_Batteriebusse\_Heide.zip}{Internet}\footnote{\href{http://lht.no-ip.biz/share/public/Technologievergleich\_Batteriebusse\_Heide.zip}{\texttt{http://lht.no-ip.biz/share/public/Technologievergleich\_Batteriebusse\_Heide.zip}}} zu finden. In den folgenden Abschnitten wird die Bedienung des Simulationsmodells erläutert.

\section{Bedienung des Simulationsmodells}
Die Simulation besteht aus mehreren Dateien, die über ihre Namen aufgerufen werden. Von daher ist das \emph{Arbeitsverzeichnis} von \textsc{Matlab} auf den Ordner \texttt{Simulationsmodell} zu setzen.

Die Simulation wird durch das Ausführen der Datei \texttt{Batteriesimulation\_Heide.m} gestartet. Durch Bearbeiten dieser Datei können die Simulationsparameter geändert werden.

Erster Parameter ist die Fahrstrecke. Hier werden ein Fahrzyklus und sein Höhendatensatz eingelesen.

In Zeile 14 kann der Präfix der Ausgabedateinamen eingestellt werden.

In den Zeilen 21-23 wird die Anzahl der Fahrzyklen zwischen Ladevorgängen sowie der minimale und maximale Ladezustand festgelegt.

Die Ladesysteme werden zwischen Zeile 26 und 48 als \texttt{array} von \texttt{structs} gespeichert. Bei Änderungen ist darauf zu achten, das die Indizierung weiterhin bei eins beginnt und der Array keine "`Löcher"' hat.

Das Ladesystem wird durch fünf Kenndaten beschrieben:
\begin{enumerate}
	\item \texttt{mass} Die Gesamtmasse der Fahrzeugseitigen Komponenten in.\\
	Einheit: kg
	\item \texttt{totzeit} Die Summe aus Zeit zwischen Halt des Busses und Ladebeginn sowie Zeit zwischen Ladeschluss und Abfahrt.\\
	Einheit: s
	\item \texttt{P} Die maximale Dauerleistung dieses Ladesystems.\\
	Einheit: W
	\item \texttt{fancyName} Der Name des Ladesystems für die Konsolenausgabe.
	\item \texttt{effizienz} Anteil der Leistung, die im Fahrzeug am Laderegler des Energiespeichers ankommt.\\
	Zahlenwert zwischen 0 und 1 (keine Prozentwerte)
\end{enumerate}

Zwischen Zeile 52 und 65 werden die Daten der Batterien eingelesen. Auch hier wird ein \texttt{array} aus \texttt{structs verwendet}. Zusätzlich zu der Dokumentation in diesem Abschnitt sollte zum Verständnis des Batteriemodells die offizielle \href{https://de.mathworks.com/help/physmod/sps/powersys/ref/battery.html}{Dokumentation}\footnote{\href{https://de.mathworks.com/help/physmod/sps/powersys/ref/battery.html}{\texttt{https://de.mathworks.com/help/physmod/sps/powersys/ref/battery.html}}} gelesen werden.

Die Batterie wird durch 19 Kenndaten beschrieben:
\begin{enumerate}
	\item \texttt{fancyName} Der Name der Batterie für die Konsolenausgabe.
	\item \texttt{uNominal} Die Spannung der Batterie am Ende des linearen Bereiches der Entladekurve.\\
	Einheit: V
	\item \texttt{qRated} Die Nennkapazität der Batterie.\\
	Einheit: Ah
	\item \texttt{qMax} Die maximale Kapazität der Batterie. Dieser Wert sollte über der Nennkapazität liegen.\\
	Einheit: Ah
	\item \texttt{uFull} Die Spannung der Batterie zu Beginn des Entladevorgangs. Dieser Wert sollte unter der Klemmenspannung ohne Last liegen.\\
	Einheit: V
	\item \texttt{iNom} Der Entladestrom, für den die Spannungs- und Kapazitätswerte angegeben sind.\\
	Einheit: A
	\item \texttt{rInternal} Der Innenwiderstand laut Datenblatt. Sollte kein Innenwiderstand dokumentiert sein, wird als Schätzwert der Zahlenwert von 1\% der Nennleistung verwendet.\\
	Einheit: $m\Omega$
	\item \texttt{comment} Ein frei verwendbares Textfeld.
	\item \texttt{qNominal} Die entnommene Ladung am Ende des linearen Bereiches.\\
	Einheit: Ah
	\item \texttt{uExponentialZone} Die Spannung zu Beginn des linearen Bereiches der Entladekurve.\\
	Einheit: V
	\item \texttt{qExponentialZone} Die entnommene Ladung zu Beginn des linearen Bereiches der Entladekurve.\\
	Einheit: Ah
	\item \texttt{ladestromgrenze} Der maximale Ladestrom für jeden Ladezustand.\\
	Format: \texttt{1x100 Double}\\
	Einheit: \texttt{A}
	\item \texttt{iMaxDischarge} Der maximale Entladestrom für einen ca. 30-sekündigen Puls.\\
	Einheit: A
	\item \texttt{mass} Die Masse einer Batterieeinheit.\\
	Einheit: $kg$
	\item \texttt{volume} Das Packvolumen einer Batterieeinheit. Bei zylindrischen Zellen Beispielsweise $\frac{2\sqrt{3}}{\pi} \cdot V_{Zelle}$.\\
	Einheit: $l$
	\item \texttt{modelToUse} Das zu verwendende Batteriemodell. Diese Variable ist derzeit ungenutzt, da nur ein Batteriemodell verwendet wird.\\
	Wert: \texttt{Ludger Heide}
	\item \texttt{BatType\_for\_simulink} Der Batterietyp, der an das \textsc{simulink}-Batteriemodell übergeben wird.\\
	Mögliche Werte: \texttt{Lead-Acid}, \texttt{Lithium-Ion}
	\item \texttt{cp} Die spezifische Wärmekapazität einer Batterie dieses Typs nach Pesaran~\cite{pesaran2001battery}.\\
	Einheit: $\frac{kJ}{kg\cdot K}$
	\item \texttt{tMax} Die höchste zulässige Batterietemperatur.\\
	Einheit: $^{\circ}C$
\end{enumerate}