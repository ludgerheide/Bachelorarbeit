\chapter*{Aufgabenstellung}
\addcontentsline{toc}{chapter}{Aufgabenstellung}

\section*{Problemstellung und Zielsetzung}
Eine der wichtigsten Fragen beim Einsatz von batteriebetriebenen Stadtbussen ist die Frage, wie der Bus aufgeladen wird und wie die Energie im Bus gespeichert wird. Inzwischen sind diverse Ladesysteme weit genug entwickelt, um im realen Betrieb eingesetzt zu werden. Parallel werden neue Forschungsergebnisse im Bereich der Batterietechnologie schnell industriell umgesetzt.\\
In dieser Bachelorarbeit werden Ladesysteme und Speichertechnologien jeweils miteinander verglichen. Die Systeme werden hierbei aus technischer Sicht betrachtet. Relevant sind zum Beispiel mechanische, elektrische und sicherheitstechnische Parameter, wirtschaftliche und politische Aspekte werden nicht explizit betrachtet. Es werden Ladesysteme verglichen, die bereits konkret existieren und im normalen Verkehrsbetrieb erprobt wurden oder werden. Auch bei den Speichertechnologien sollen Technologien verglichen werden, die jetzt oder in naher Zukunft im industriellen Maßstab produziert werden können.\\
Ziel ist es, die Effizienz verschiedener Kombinationen von Ladetechnologie, Speichertechnologie und Ladestrategie auf verschiedenen Linien vergleichen zu können.

\section*{Grundsätzliche Vorgehensweise}
\begin{enumerate}
	\item Recherche über existierende Ladesysteme und Speichertechnologien für Elektrobusse
	\item systematische Bestimmung der technischen Parameter, nach denen die Eignung der verschiedenen Ladesystem und Speichertechnologien beurteilt wird
	\item Modellrechnungen verschiedener Technologiekombinationen
	\item Ermittlung der besten Technologiekombination für eine bestimmte Strecke und Diskussion der Ergebnisse
\end{enumerate}