\chapter*{Aufgabenstellung}
\addcontentsline{toc}{chapter}{Aufgabenstellung}

\section*{Problemstellung und Zielsetzung}
Zwei der wichtigsten Fragen beim Einsatz von batteriebetriebenen Stadtbussen sind, wie der Bus aufgeladen wird und wie die Energie im Bus gespeichert wird. Inzwischen sind diverse Ladesysteme weit genug entwickelt, um im praktischen Betrieb eingesetzt zu werden. Praktisch einsetzbare Speichertechnologien existieren bereits und laufend werden neue Forschungsergebnisse im Bereich der Batterietechnologien schnell industriell umgesetzt.\\
In dieser Bachelorarbeit werden verschiedene Ladesysteme und Speichertechnologien verglichen. Die Systeme werden hierbei aus technischer Sicht betrachtet, zum Beispiel unter Berücksichtigung mechanischer, elektrischer und sicherheitstechnischer Parameter. Wirtschaftliche und politische Aspekte werden nicht explizit betrachtet. Es werden Ladesysteme verglichen, die bereits im normalen Verkehrsbetrieb erprobt wurden oder werden. Auch bei den Speichertechnologien werden nur Technologien verglichen, die jetzt oder in naher Zukunft im industriellen Maßstab produziert werden können.\\
Ziel ist es, die Effizienz verschiedener Kombinationen von Ladetechnologie, Speichertechnologie und Ladestrategie auf verschiedenen Buslinien vergleichen zu können.

\section*{Grundsätzliche Vorgehensweise}
\begin{enumerate}
	\item Überblick über existierende Ladesysteme und Speichertechnologien für Elektrobusse.
	\item Systematische Festlegung der technischen Parameter, nach denen die Eignung der verschiedenen Ladesysteme und Speichertechnologien beurteilt wird.
	\item Modellrechnungen zur Ermittlung der für jede Kombination von Ladesystem und Speichertechnologie unterschiedlichen Werte.
	\item Ermittlung der besten Technologiekombination für zwei ausgesuchte Strecken und Diskussion der Ergebnisse.
\end{enumerate}