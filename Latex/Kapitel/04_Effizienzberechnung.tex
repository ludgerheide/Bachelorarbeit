\chapter{Bewertung} %TODO: Besserer Titel %Drittens
Ziel dieser Arbeit soll der Vergleich verschiedener Kombinationen von Ladesystem und Speichertechnologie aus technischer Sicht ein. Qualitative Bewertungen einer Kombination sind teilweise ohne Rechnung möglich (z. B. "`\textsc{Primove} ist besser für Lithium-Titanat-Batterien geeignet als für Lithium-Eisenphosphat-Batterien"'). Eine genauere Betrachtung und ein Vergleich von ähnlichen Technologiekombinationen ist jedoch nicht möglich. Daher werden in dieser Arbeit mit Hilfe eines Simulationsmodell quantitative Daten zu den verschiedenen Technologiekombinationen ermittelt.

\section{Simulationsmodell}
\marginpar{Gibt es dazu ne Quelle?}

\section{Eingabewerte}
Als Eingabewerte werden die in Tabellen\marginpar{Link} xxx und xxx eingeführten Daten verwendet. Als Fahrzyklus wird der "`Berlin City Cycle"' gewählt\marginpar{Quelle!}.

Bei Systemen mit Gelegenheitsladung wird nach jedem\marginpar{jedem?} Fahrzyklus die Batterie soweit geladen, dass der Fahrzyklus zwei Mal durchfahren werden kann\marginpar{warum 2mal?}.

\section{Ausgabedaten der Simulation}
\begin{description}
	\item[Energieverbrauch] Die aus dem Stromnetz entnommene Energie.\\
	Einheit: $\frac{kWh}{km}$
	\item[Batteriegewicht] Das erforderliche Gewicht der Batterie 	
\end{description}

\section{Ladesystem}
\section{Speichertechnologie}
\section{Ladestrategie und Route}
\section{Ergebnisse}
\subsection{Route A}
\subsection{Route B}