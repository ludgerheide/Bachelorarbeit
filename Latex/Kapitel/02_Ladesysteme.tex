\chapter{Ladesysteme}
\section{Bewertungskriterien} \marginpar{Subsections}
\begin{itemize}
	\item Mechanik \marginpar{Bessere Beschreibung (Zeit ist nicht mechanisch)}
	\begin{description}
		\item \textbf{Abmaße, feste Position} \emph{Fahrzeug- uns Stationsseitig}\\
		Die Abmaße jener Komponenten des Ladesystems, deren Postion relativ zu ihrem Gegenüber fest ist.\\
		Beispiel: Pantograph und Stromschiene
		\item \textbf{Abmaße, freie Position} \emph{Fahrzeug- uns Stationsseitig}\\
		Die Abmaße jener Komponenten des Ladesystems, deren Postion frei gewählt werden kann.\\
		Beispiel: Elektrische Wandler
		\item \textbf{Masse} \emph{Fahrzeugseitig}\\
		Gesamtmasse der fahrzeugseitigen Komponenten
		\item \textbf{Freiheitsgrade} \emph{Fahrzeug- uns Stationsseitig}\\
		Anzahl der Freiheitsgrade, die eine bewegliche Komponente relativ zum Fahrzeug oder zur Station hat. 0, wenn es keine beweglichen Teile gibt.
		\item \textbf{Positionierungsgenauigkeit des Busses} \emph{Fahrzeugseitig} \\
		Eforderliche Genaugkeit der Positionierung des Busses relativ zur Ladestation, so dass noch 80\% der Standardladeleistung verfügbar ist. Besteht aus zwei translatorischen und zwei rotatorischen Komponenten (X, Y, Richtung des Busses, Kneelingwinkel)
		\item \textbf{Totzeit}\\
		Summe aus Zeit zwischen Halt des Busses und Ladebeginn und Zeit zwischen Ladeschluss und Abfahrt. Ist die Zeit von der Position abhängig, so wird jeweils eine Abweichung um die Hälfte des maximalen Wertes angenommen.
	\end{description}
	\item Elektrik
	\begin{description}
		\item \textbf{Anzahl der Wandler} \emph{Fahrzeug- und Stationsseitig}\\
		Anzahl der Komponenten, die den Spannungsverlauf verändern (Gleichrichter, Transformatoren etc.)
		\item \textbf{Spannung}\\
		Mit welcher Spannung wird geladen? Gleich- oder Wechselspannung? Nicht relevant bei induktiver Ladung und Batteriewechselsystemen.
		\item \textbf{Leistung}\\
		Aus dem öffentlichen Stromnetz erforderliche Leistung. Die im Bus angekommene Leitung ergibt sich durch die Effizienz. Es wird angenommen, das die Leistung aus dem ortsüblichen Niederspannungsnetz kommt. (400V Drehstrom in Europa)\marginpar{Amerika}
		\item \textbf{Effizienz}\\
		Prozentualer Anteil der Leistung, die im Fahrzeug am Laderegler ankommt.		
	\end{description}
	\item Sicherheit
	\begin{description}
		\item \textbf{Fehlerstromschutz}\\
		Wie wird vor Fehlerströmen geschützt?\\
		Beispiel: Fahrzeugseitige Isolationsüberwachung, RCD Typ B in Ladestation
		\item \textbf{Exponierte Leiter}\\
		Sind die spannungsführenden Leiter nur durch die umgebende Luft isoliert? Mögliche Antworten: nie, nur beim Ladevorgang, immer
		\item \textbf{Zugänglichkeit der Schnittstelle}\\
		Wo am Fahrzeug ist die Ladeschnittstelle? Wie ist sie geschützt?
		\item \textbf{Strahlung – Gesundheitsaspekte}\\
		\marginpar{Wie beschriebt man das? Macht es Sinn?}
		\item \textbf{Strahlung – EMV-Aspekte}\\
		\marginpar{Wie beschriebt man das? Macht es Sinn?}
		\item \textbf{Position der Ladestation}\\
		Ladesäule oder Unterflur? Im Depot oder an öffentlich zugänglicher Station?
	\end{description}
	\item Reife
	\begin{description}
		\item \textbf{Fahrzeugkilometer – Teststrecke}\\
		Mit diesem Ladesystem zurückgelegte Fahrzeugkilometer außerhalb des normalen Verkehrs.
		\item \textbf{Fahrzeugkilometer – Verkehr}\\
		Mit diesem Ladesystem im normalen Verkehr zurückgelegte Fahrzeugkilometer.
		\item \textbf{Erfahrungen der Vekehrsgesellschaften}\\
		War das System zuverlässig oder fehleranfällig? Hat sich die Zuverlässigkeit während des Betriebes verbessert?\marginpar{Macht es Sinn?}
	\end{description}
\end{itemize}
\section{Betrachtete Systeme} %TODO: Subsections
\section{Vergleichstabelle}