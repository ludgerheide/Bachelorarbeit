\chapter*{Abstract}
\addcontentsline{toc}{chapter}{Abstract}
\section*{English}
\selectlanguage{english}
Using a simulation model the technical characteristics of different combinations of charging systems and energy storage systems (= batteries) were computed for two different bus routes. Input data for the simulation were the mechanical and electrical parameters of the charging systems and of the batteries, the mechanical parameters of the bus as well as a recording of a regular passenger transport trip on the respective bus route. For each combination of charging system and energy storage system, the simulation computed the smallest battery required to operate this bus route under defined conditions (e.g. overnight charging versus opportunity charging). For every combination, the output parameters battery size, energy consumption, battery cooling requirement and charging time were calculated.

Based on the simulation results subsequently a weighted assessment was carried out according to VDI 2225 to compare the technical suitability of each combination. Significant differences were observed in the suitability of different combinations of technologies.

Using the methods developed in this study, transportation agencies and policy makers can obtain reliable data on the technical characteristics of different battery-powered buses in order to make a well-informed decision for a specific combination of technologies for charging and for energy storage.


\section*{Deutsch}
\selectlanguage{ngerman}
Mit Hilfe eines Simulationsmodells werden technische Charakteristika von verschiedenen Kombinationen von Ladesystem und Speichertechnologien für zwei unterschiedliche Streckenführungen (Buslinien) berechnet. Eingangsdaten der Simulation sind mechanische und elektrische Parameter von verschiedenen Ladesystemen und Speichertechnologien, die mechanischen Parameter des Busses sowie die Aufzeichnung einer Linienverkehrs-Fahrt auf der jeweiligen Buslinie. Die Simulation berechnet für jede Kombination von Ladesystem und Speichertechnologie die kleinste Batterie, mit der diese Buslinie unter definierbaren Randbedingungen (z.B. Gelegenheits- oder Nachtladung) gefahren werden kann. Für jede Kombination werden Batteriegröße, Energieverbrauch, Batteriekühlung und Ladezeit berechnet.

Auf Basis der Simulationsergebnisse wird anschließend eine Bewertung nach VDI2225 durchgeführt, um die technische Eignung der verschiedenen Kombinationen zu vergleichen. Es zeigen sich deutliche Unterschiede in der Eignung verschiedener Technologiekombinationen.

Mit der in dieser Arbeit vorgestellten Methode können Verkehrsbetriebe sowie politische und wirtschaftliche Entscheidungsträger zuverlässige Daten über die technischen Charakteristika von verschiedenen batteriebetriebenen Bussen erhalten und auf dieser Grundlage eine fundierte Entscheidung für eine bestimmte Technologiekombination treffen.