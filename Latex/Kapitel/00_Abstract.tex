\chapter*{Abstract}
\addcontentsline{toc}{chapter}{Abstract}
\section*{English}
\selectlanguage{english}
The behavior of multiple combinations of charging system and battery on two different bus lines is calculated using a simulation model. The techological quality of each combnation is evaluated using a weighted assessment.

The input values for the simulation are mechanical and electrical paramters of each charging system and battery as well as mechanical paramters of the bus and a recording of speeds and gradients along the bus line. The simulation calculates the smallest possible battery to operate this bus line for each combination of battery and charging system. Battery size, energy consumption, battery cooling and charging time are calculated for each combination.

The simulation results are normalized to point values and a weighted average is calculated in accordance with VDI 2225.

Using the method presented here transportation agencies and policy makers can obtain reliable data about the technological characteristics of different battery-driven buses in order to make an informed decision for a specific technology combination.

\section*{Deutsch}
\selectlanguage{ngerman}
Mit Hilfe eines Simulationsmodells wurde das Verhalten von verschiedenen Kombinationen von Ladesystem und Speichertechnologie auf zwei unterschiedlichen Buslinien berechnet. Die technische Qualität der verschiedenen Kombinationen wurde durch eine gewichtete Bewertung ermittelt.

Eingangsdaten der Simulation sind mechanische und elektrische Parameter von Ladesystemen und Speichertechnologien, die mechanischen Parameter des Busses sowie die Aufzeichnung eines Umlaufs auf der gewünschten Buslinie. Die Simulation berechnet für jede Kombination von Ladesystem und Speichertechnologie die kleinste Batterie, mit der dieser Umlauf unter definierbaren Randbedingungen gefahren werden kann. Für jede Kombination werden Batteriegröße, Energieverbrauch, Batteriekühlung und Ladezeit berechnet.

Auf Basis der Simulationsergebnisse wird eine technische Bewertung nach VDI 2225 durchgeführt, um die technische Wertigkeit jeder Kombination zu ermitteln.

Mit der in dieser Arbeit vorgestellten Methode können Verkehrsbetriebe sowie politische und wirtschaftliche Entscheidungsträger zuverlässige Daten über die technischen Charakteristika eines batteriebetriebenen Busses erhalten und so eine fundierte Entscheidung für eine bestimmte Technologiekombination treffen.