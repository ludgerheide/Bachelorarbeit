\chapter*{Abstract}
\addcontentsline{toc}{chapter}{Abstract}
\section*{English}
\selectlanguage{english}
The behavior of several combinations of charging system and is calculated on two different bus lines using a simulation model. The input values for the simulation are the mechanical and electrical parameters of each charging system and battery, the mechanical parameters of the bus as well as a recording of one tour on the respective bus line. The simulation calculates the smallest possible battery to operate this bus line under user-definable constraints (e.g. overnight or opportunity charging), for each combination of battery and charging system. Battery size, energy consumption, battery cooling and charging time are calculated for each combination.

To assess the technological suitability of each combination, a weighted assessment according to VDI 2225 is performed on the simulation results. Significant differences in the suitability of different combinations were observed.

Using the method presented here transportation agencies and policy makers can obtain reliable data about the technological characteristics of different battery-driven buses in order to make an informed decision for a specific technology combination.

\section*{Deutsch}
\selectlanguage{ngerman}
Mit Hilfe eines Simulationsmodells werden technische Charakteristika von verschiedenen Kombinationen von Ladesystem und Speichertechnologien für zwei unterschiedliche Streckenführungen (Buslinien) berechnet. Eingangsdaten der Simulation sind mechanische und elektrische Parameter von verschiedenen Ladesystemen und Speichertechnologien, die mechanischen Parameter des Busses sowie die Aufzeichnung einer Linienverkehrs-Fahrt auf der jeweiligen Buslinie. Die Simulation berechnet für jede Kombination von Ladesystem und Speichertechnologie die kleinste Batterie, mit der diese Buslinie unter definierbaren Randbedingungen (z.B. Gelegenheits- oder Nachtladung) gefahren werden kann. Für jede Kombination werden Batteriegröße, Energieverbrauch, Batteriekühlung und Ladezeit berechnet.

Auf Basis der Simulationsergebnisse wird anschließend eine Bewertung nach VDI2225 durchgeführt, um die technische Eignung der verschiedenen Kombinationen zu vergleichen. Es zeigen sich deutliche Unterschiede in der Eignung verschiedener Technologiekombinationen.

Mit der in dieser Arbeit vorgestellten Methode können Verkehrsbetriebe sowie politische und wirtschaftliche Entscheidungsträger zuverlässige Daten über die technischen Charakteristika von verschiedenen batteriebetriebenen Bussen erhalten und auf dieser Grundlage eine fundierte Entscheidung für eine bestimmte Technologiekombination treffen.